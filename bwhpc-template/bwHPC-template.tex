%%%%% 10. bwHPC Symposium 2024 %%%%%

%%% START: Mandatory header %%%
% Turn on warnings because of bad style
\RequirePackage[l2tabu,orthodox]{nag}

% Document class scrartcl
\documentclass[
  paper       = a4,
  headheight  = 16pt,
  footheight  = 16pt,
  fontsize    = 10pt,
  twoside     = true,
  titlepage   = true,
]{scrartcl}

% Postscript type 1 fonts
% For Xelatex/Lualatex
% \usepackage{fontspec}
% For pdflatex
\usepackage[utf8]{inputenc}

% Language settings, please select one
% English
\usepackage[ngerman,british]{babel}
\usepackage[english=british]{csquotes}
\setquotestyle[guillemets]{german}

% or German
% \usepackage[english,ngerman]{babel}
\usepackage[german=guillemets]{csquotes}


% LaTeX bibliography style, please select one
% biblatex with bibtex fallback (old)
\input{../bwhpc-layout/bibtex.tex}
% Use modern biblatex + biber -> New Texlive or Overleaf
% See url for setup with your favorite editor:
% https://tex.stackexchange.com/questions/154751/biblatex-with-biber-configuring-my-editor-to-avoid-undefined-citations
% \usepackage[
  backend=biber,
  natbib=true,
  citestyle=authoryear,
  style=authoryear,
  maxcitenames=1,
  maxbibnames=5,
  uniquename=init,
  giveninits=true,
  sortlocale={de_DE_phonebook},
]{biblatex}

\renewcommand*{\bibfont}{\small}

\DefineBibliographyExtras{british}{%
 %from german.lbx
  \protected\def\mkbibdatelong#1#2#3{%
    \iffieldundef{#3}
      {}
      {\mkbibordinal{\thefield{#3}}%
       \iffieldundef{#2}{}{\nobreakspace}}%
    \iffieldundef{#2}
      {}
      {\mkbibmonth{\thefield{#2}}%
       \iffieldundef{#1}{}{\space}}%
    \iffieldbibstring{#1}
      {\bibstring{\thefield{#1}}}
      {\dateeraprintpre{#1}\stripzeros{\thefield{#1}}}}%
  \protected\def\mkbibdateshort#1#2#3{%
    \iffieldundef{#3}
      {}
      {\mkdayzeros{\thefield{#3}}\adddot
       \iffieldundef{#2}{}{\thinspace}}%
    \iffieldundef{#2}
      {}
      {\mkmonthzeros{\thefield{#2}}%
       \iffieldundef{#1}
     {}
     {\iffieldundef{#3}{/}{\adddot\thinspace}}}%
    \iffieldbibstring{#1}
      {\bibstring{\thefield{#1}}}
      {\dateeraprintpre{#1}\mkyearzeros{\thefield{#1}}}}%
}

% use eprint for urn: example fields (bib-file)
% % eprint = {urn:nbn:de:bsz:100-opus-5191},
% % eprinttype = {urn},
% in your document preamble define
\DeclareFieldFormat{eprint:urn}{%
  \textsc{\scriptsize{URN\addcolon}}\space
  \ifhyperref
    {\href{http://www.nbn-resolving.org/#1}{\nolinkurl{#1}}}
    {\nolinkurl{#1}}
}

\DeclareFieldFormat{eprint:hdl}{%
  \textsc{\scriptsize{HDL\addcolon}}\space
  \ifhyperref
    {\href{http://hdl.handle.net/#1}{\nolinkurl{#1}}}
    {\nolinkurl{#1}}
}

\bibsetup{
  \setcounter{abbrvpenalty}{10000}
  \setcounter{highnamepenalty}{10000}
  \setcounter{lownamepenalty}{10000}
}

\AtEveryBibitem{\clearfield{month}}
\AtEveryBibitem{\clearfield{day}}

% Change this to your bib file
\addbibresource{literature.bib}

% Include bwHPC-Style formatting at the END
% Resize paper size to print format
% \setlength\marginparwidth{0pt}
\usepackage[%
  paperwidth  = 17cm,
  paperheight = 24cm,
  top         = 25mm,
  bottom      = 25mm,
  outer       = 15mm,
  inner       = 23mm,
  footskip    = 10mm,
  heightrounded,
  % showframe,
]{geometry}
% space between text and footnote
\setlength{\skip\footins}{20pt}

% xcolors
\usepackage{xcolor}

% don't fill page on two-sized pages
\raggedbottom

% Line spacing
\usepackage{setspace}
\setstretch{1,3}

% Hurenkinder und Schusterjungen
% Hurenkinder
\widowpenalty = 10000
\displaywidowpenalty = 10000
% Schusterjungen
% \clubpenalty = 10000
% Hurenkinder Alternativen
% \looseness=-1 immediately after the last word of the paragraph to set it tight
% \enlargethispage{\baselineskip} to add a line to the page
% % \enlargethispage{-\baselineskip} may produce a (more-or-less) acceptable two-line widow

% DRAFT
% \usepackage{draftwatermark}
% \SetWatermarkText{\textsc{DRAFT}}
% \SetWatermarkColor[gray]{0.9}

% Floating figures
\usepackage{graphicx}
\usepackage{subfig}

% Postscript type 1 fonts
% For Xelatex/Lualatex
%\usepackage{fontspec}
% For pdflatex
%\usepackage[utf8]{inputenc}
% Font charter or utopia
% \usepackage{utopia}
\renewcommand*{\familydefault}{\rmdefault}

% for \FloatBarrier
\usepackage{placeins}

% math packages
\usepackage{mathtools}
\usepackage{amsfonts}
\usepackage{amssymb}

% Caption setup
\usepackage[font={footnotesize}]{caption}
\setcapwidth{0.8\textwidth}
\setcapindent{0pt}
\addtokomafont{caption}{\centering}
\addtokomafont{captionlabel}{\bfseries}

% colors for listings
\colorlet{commentcolour}{green!50!black}
\colorlet{stringcolour}{red!60!black}
\colorlet{keywordcolour}{magenta!90!black}
\colorlet{exceptioncolour}{yellow!50!red}
\colorlet{commandcolour}{blue!60!black}
\colorlet{numpycolour}{blue!60!green}
\colorlet{literatecolour}{magenta!90!black}
\colorlet{promptcolour}{green!50!black}
\colorlet{specmethodcolour}{violet}

% Listings for code
\usepackage{textcomp} % enable straight quotes in listings package
\usepackage{listings}
\lstset{
  upquote           = true, % straight quotes
  captionpos        = b, % bottom
  frame             = shadowbox,
  framesep          = 5pt,
  framexleftmargin  = -5pt,
  framexrightmargin = -10pt,
  rulecolor         = \color{black!40},
  rulesepcolor      = \color{lightgray!60},
  aboveskip         = \bigskipamount,
  numberstyle       = {\scriptsize},
  numbersep         = -8pt,
  % numbers           = left
}
\makeatletter
\lstdefinestyle{mylststyle}{
  basicstyle=%
    \ttfamily
    \lst@ifdisplaystyle\footnotesize\fi
}
\makeatother
\lstset{style=mylststyle}
% Python highlighting
\lstset{
stringstyle=\color{stringcolour},
alsoletter={1234567890},
otherkeywords={\%, \}, \{, \&, \|},
keywordstyle=\color{keywordcolour}\bfseries,
emph={and,break,class,continue,def,yield,del,elif ,else,%
except,exec,finally,for,from,global,if,import,in,%
lambda,not,or,pass,print,raise,return,try,while,assert,with},
emphstyle=\color{blue}\bfseries,
emph={[2]True, False, None},
emphstyle=[2]\color{keywordcolour},
emph={[3]object,type,isinstance,copy,deepcopy,zip,enumerate,reversed,list,set,len,dict,tuple,xrange,append,execfile,real,imag,reduce,str,repr},
emphstyle=[3]\color{commandcolour},
emph={Exception,NameError,IndexError,SyntaxError,TypeError,ValueError,OverflowError,ZeroDivisionError},
emphstyle=\color{exceptioncolour}\bfseries,
morecomment=[s]{"""}{"""},
commentstyle=\color{commentcolour}\slshape,
emph={[4]ode,fsolve,sqrt,exp,sin,cos,arctan,arctan2,arccos,pi,array,norm,solve,dot,arange,isscalar,max,sum,flatten,shape,reshape,find,any,all,abs,plot,linspace,legend,quad,polyval,polyfit,hstack,concatenate,vstack,column_stack,empty,zeros,ones,rand,vander,grid,pcolor,eig,eigs,eigvals,svd,qr,tan,det,logspace,roll,min,mean,cumsum,cumprod,diff,vectorize,lstsq,cla,eye,xlabel,ylabel,squeeze},
emphstyle=[4]\color{numpycolour},
emph={[5]__init__,__add__,__mul__,__div__,__sub__,__call__,__getitem__,__setitem__,__eq__,__ne__,__nonzero__,__rmul__,__radd__,__repr__,__str__,__get__,__truediv__,__pow__,__name__,__future__,__all__},
emphstyle=[5]\color{specmethodcolour},
emph={[6]assert,yield},
emphstyle=[6]\color{keywordcolour}\bfseries,
emph={[7]range},
emphstyle={[7]\color{keywordcolour}\bfseries},
}

% hyperref package
% change font to rmfamily and/or itshape
\renewcommand\UrlFont{\rmfamily}
\usepackage[%
  colorlinks  = true,%
  allcolors   = black,%
  urlcolor    = black,%
  pdfencoding = auto,%
  hyperindex,%
  pageanchor,%
  pdfdisplaydoctitle,%
  psdextra,%
  frenchlinks=true,
  final%
]{hyperref}
\usepackage{doi}

% Header/Footer
\usepackage[headsepline]{scrlayer-scrpage}
\clearpairofpagestyles
\setkomafont{pageheadfoot}{\footnotesize \rmfamily}
\setkomafont{pagination}{\footnotesize \rmfamily}
% KSP narrow spacing
\setkomafont{title}{\linespread{1} \huge \rmfamily \textbf}
\setkomafont{subtitle}{\linespread{1} \large \rmfamily \textbf}
\setkomafont{author}{{\linespread{1} \rmfamily}}
\setkomafont{date}{\rmfamily}
\setkomafont{dedication}{{\linespread{1} \rmfamily}}
\setkomafont{publishers}{\rmfamily}
\setkomafont{subject}{\rmfamily}
\setkomafont{titlehead}{\rmfamily}
% KSP narrow spacing
\addtokomafont{disposition}{\linespread{1} \rmfamily}

% Add titlehead to titlepage
\DeclareNewLayer[%
  headsep,
  foreground,
  contents={%
    \begin{minipage}[t]{\textwidth}%
      \usekomafont{titlehead}{\csname @titlehead\endcsname\par}%
    \end{minipage}\par}
]{titlehead}
\DeclarePageStyleByLayers{titlepage}{titlehead}
\AddToHook{cmd/maketitle/after}{\thispagestyle{titlepage}}

% Abstract neu definieren
% \makeatletter
% \@ifclassloaded{scrbook}{}{
%   \renewenvironment{abstract}
%     {\normalsize
%       \list{}{
% 	% \setlength{\leftmargin}{0.05\textwidth}%
%     \setlength{\leftmargin}{0\textwidth}%
% 	\setlength{\rightmargin}{\leftmargin}%
%       }%
%       \item\relax}
%       {\endlist}
%   }
% \makeatother

% Remove parskip on items
\makeatletter
\def\@listI{\leftmargin\leftmargin
            \parsep 1\p@ \@plus2\p@ \@minus\p@
            \topsep 8\p@ \@plus2\p@ \@minus4\p@
%             \topsep\z@
            \itemsep1\p@ \@plus2\p@ \@minus\p@}
\makeatother

\newcommand*{\mainauthor}[1]{\def\MainAuthor{#1}}
\newcommand*{\cauthor}[1]{\def\CAuthor{#1}}
\newcommand*{\booktitle}{Proceedings of the 10th bwHPC Symposium}
\newcommand*{\headtitle}[1]{\def\HeadTitle{#1}}
\newcommand*{\maintitle}[1]{\def\MainTitle{#1}}
\newcommand*{\subt}[1]{\def\SubTitle{#1}}
\newcommand*{\fulltitle}[1]{\def\FullTitle{#1}}
\newcommand*{\orcidauth}[1]{\def\OrcID{#1}}
\newcommand*{\orcid}[1]{\href{#1}{\includegraphics[height=10pt]{../bwhpc-layout/orcidlogo.pdf}\,\nolinkurl{#1}}}
\newcommand*{\orcidlogo}[1]{\href{#1}{\includegraphics[height=10pt]{../bwhpc-layout/orcidlogo.pdf}}}
\newcommand*{\affiliation}[1]{\date{#1}}
\newcommand*{\cmail}[1]{\def\CMail{#1}}
\newcommand*{\email}[1]{\href{mailto:#1}{\nolinkurl{#1}}}
\newcommand*{\docver}[1]{\def\DocVer{#1}}
\newcommand*{\pagestart}[1]{\def\Page{#1}}
\newcommand*{\keywords}[1]{\def\Keywords{#1}}

%%% page config

% footer
\ofoot[]{\pagemark}

% delete date
\date{}

% Update PDF info
\hypersetup{
  pdfsubject  = {Proceedings of the 10th bwHPC Symposium, Freiburg September 2024},
  pdfkeywords = {bwHPC, HPC, 10th bwHPC Symposium},
}


%%% END: Mandatory header %%%

%%% START: Mandatory document configuration %%%

% add author info, please add main author to author.tex
%FILE=bwHPC2024-MJ-bwForCluster-NEMO
% Meta information for PDF, etc.
% Main author, for contract, etc. not corresponding author
\mainauthor{Michael Janczyk}
% Main title
\maintitle{My Paper for the 10th bwHPC Symposium}
% Subtitle, if available
\subt{Freiburg September 2024}
% Full title for PDF information (select one)
% Using subtitle
\fulltitle{\MainTitle{ }--{ }\SubTitle}
% No subtitle
%\fulltitle{\MainTitle}
% Document keywords
\keywords{{bwHPC}, {HPC}, {bwHPC Symposium}}
% DO NOT EDIT!
\headtitle{\MainTitle}
% Will be added when published
\pagestart{1}


% Start document
\begin{document}

% Titlepage
\begin{titlepage}
% DO NOT CHANGE following 4 lines
\thispagestyle{myheadings}
\raggedright
\vspace*{.05\textheight}

% Title, will be read from author.tex
{\usekomafont{title}
\MainTitle % will be read from author.tex, or change here directly
\par}

% Remove following 4 lines if no sublitle available
\vspace{\baselineskip}
{\usekomafont{subtitle}
\SubTitle{} % will be read from author.tex, or change here directly
\par}

% DO NOT CHANGE following line
\vspace{\baselineskip}

% AUTHORS
% Use when all authors have same affilaition, else see below
% Authors
{\usekomafont{author}
% Corresponding author and ORCID, see last sections of this document
% \hyperref[sec:cauthor]{AUTHOR} is an anchor to the corresponding author, please change if not main author
% One author per line
% Corresponding author info at the end
\hyperref[sec:cauthor]{\MainAuthor}% will be read from author.tex, or change here directly
    \,\orcidlogo{https://orcid.org/0000-0003-4886-736X},
Dirk von Suchodoletz%
    \,\orcidlogo{https://orcid.org/0000-0002-4382-5104},
Bernd Wiebelt%
    \,\orcidlogo{https://orcid.org/0000-0003-2771-4524}
\par}

% Affiliation
\vspace{.5\baselineskip}
{\usekomafont{author}
% Change affiliation
\small
eScience Abteilung, Albert-Ludwigs-Universität, Freiburg, Deutschland
\par}

% % For multiple affiliations use markers like *, dag, etc.
% % Authors
% {\usekomafont{author}
% % Corresponding author and ORCID, see last sections of this document
% % \hyperref[sec:cauthor]{AUTHOR} is an anchor to the corresponding author, please change if not main author
% % One author per line
% % Corresponding author info at the end
% \hyperref[sec:cauthor]{\MainAuthor}\textsuperscript{*},
% Bernd Wiebelt\textsuperscript{*},
% Dirk von Suchodoletz\textsuperscript{*},
% Bernd Brot\textsuperscript{\dag}
% \par}

% % Affiliation
% \vspace{.5\baselineskip}
% {\usekomafont{author}
% % Change affiliation
% \small
% % Affiliation 1
% \textsuperscript{*}{eScience, Universität Freiburg, Freiburg i. Br., Deutschland
% \\
% \small
% % Affiliation 2
% \textsuperscript{\dag}Kika, 99081 Erfurt, Germany
% \par}

% DO NOT CHANGE following 4 lines
\vspace{\baselineskip}
\let\endtitlepage\relax
\end{titlepage}
% Titlepage end

% %%% END: Mandatory document configuration %%%

% %%% START: Header configuration and PDF title %%%

% Define header
\rohead{\HeadTitle} % will be read from author.tex, or change here directly
\lehead{\MainAuthor~et~al.} % will be read from author.tex, or change here directly

% Write PDF information
\hypersetup{
  pdftitle={\FullTitle},
  pdfauthor={\MainAuthor~et~al.},
  pdfkeywords={\Keywords},
}

% Set page for paper
\setcounter{page}{\Page}

%%% END: Header configuration and PDF title %%%


%%% Start: Document %%%

% Select EN or DE
% English
\section*{Abstract}
% or German
%\section*{Kurzfassung}
This is a style guide for contributions to the bwHPC Symposium 2024 in Freiburg.
Please read carefully.


% Section 1
\section{Formatting}

The following sections define the layout specifically for contributions to the bwHPC Symposium 2024 in Freiburg.
Please read them carefully.

If your code does not compile or you have errors or other questions, please send us the file or ask us how to solve the problem.
We will correct and reformat all submissions before publication.


% Section 1.1
\subsection{Language Settings}

You can use language-specific formatting, except for dates.


% Section 1.1.1
\subsubsection{Dates}

Dates should be formatted as dd.mm.yyyy or yyyy-mm-dd (e.g. 31.01.2018, 2018-01-31) unless they are code, shell output or filenames.
Unfortunately, 31/01/2018 (UK) and 01/31/2018 (US) are ambiguous e.g. 01/02/2018.


% Section 1.1.2
\subsubsection{Quotes}

This text is \enquote{placed in quotation \enquote{marks automatically}} with \enquote{csquotes}.\footnote{csquotes: \url{https://ctan.org/pkg/csquotes}, visited on 31.01.2024.}
You can force \enquote*{inner/single} quotation marks at any time.
Remember to always \enquote{end your quote}, otherwise the file will not compile.


% Section 1.1.2.1
\paragraph{German Umlauts.}

German umlauts can either be activated by adding \lstinline|\"| in front of the letter \"a or by using \verb|\usepackage[utf8]{inputenc}| (already activated in \lstinline|layout.tex|).
Then you can simply write \enquote{ä}.

For UtF-8 we will probably use \enquote{xelatex} with \verb|\usepackage{fontspec}| for the final document.
You are welcome to adjust the setting yourself in \lstinline|layout.tex|.
In Overleaf, you can switch from \enquote{pdflatex} to \enquote{xelatex} in the document settings.


% Section 1.2
\subsection{Citation}

For citations, use Bibtex or Biber. Biblatex with Bibtex8 backend is the default for this document, but consider using Biblatex and Biber (see header for more information). Insert your Bibtex/Biber file with \verb|\addbibresource{}|. Example:

\verb|\addbibresource{literature.bib}|.

The final document will use Biber, which offers many more options.
If you want to try it out, you should first convert your bib file to Biblatex, e.g. with Jabref.

Use \verb|\parencite{<ref>}| for citations.\footnote{Biblatex: \url{https://ctan.org/pkg/biblatex}, visited on 31.01.2024.}
In the \verb|bwhpc-template| directory you will find a sample file for a Biblatex file with the publication \enquote{bwForCluster NEMO. Forschungscluster für die Wissenschaft}\parencite{bwhpc2019:janczyk_bwforcluster}.


\subsection{Units}

For units, use the small/half space \lstinline|'10\,GB'|, e.g. 10\,GB.
However, you can also use packages such as \enquote{units},\footnote{units: \url{https://ctan.org/pkg/units}, visited on 31.01.2024}
or \enquote{siunitx}.\footnote{siunitx: \url{https://ctan.org/pkg/siunitx}, visited on 31.01.2024}
Please use SI units/prefixes.\footnote{Metric Prefixes: \url{https://en.wikipedia.org/wiki/Metric_prefix}, visited on 31.01.2024}


\subsection{Sections}

The following sections are available for \enquote{scrartcl}.
Sections do not have end punctuation.
Paragraphs must be punctuated at the end.

\verb|\section|, \verb|\subsection|, \verb|\subsubsection| and \verb|\paragraph|, \verb|\subparagraph|, example:
\begin{verbatim}
\section{This is a First-Order Title}
\subsection{This is a Second-Order Title}
\subsubsection{This is a Third-Order Title}
\paragraph{This is a Fourth-Order Title.}
\subparagraph{This is a Fifth-Order Title.}
\end{verbatim}


\subsection{Listings and Code}

For listings and code, we recommend using \enquote{lstlisting}.\footnote{Listings on Overleaf: \url{https://www.overleaf.com/learn/latex/Code_listing}, visited on 31.01.2024}
\begin{lstlisting}[language=Python, caption=Implementation of ..., label=lst:numpy, numbers=left]
    import numpy as np
    ...
\end{lstlisting}


\section{Compiling \LaTeX}

For \enquote{classic} Pdflatex and Bibtex, you can try to compile your files with the following commands:
\begin{lstlisting}[language=sh, caption=Build Document with Pdflatex and Bibtex, label=lst:pdflatex]
    pdflatex bwHPC-template.tex
    bibtex bwHPC-template
    pdflatex bwHPC-template.tex
    pdflatex bwHPC-template.tex
\end{lstlisting}

For Xelatex/Lualatex and Biber you can try to compile your files with the following commands (\autoref{lst:xelatex}):
\begin{lstlisting}[language=sh, caption=Build Document with Xelatex and Biber, label=lst:xelatex]
    xelatex bwHPC-template.tex
    biber bwHPC-template
    xelatex bwHPC-template.tex
    xelatex bwHPC-template.tex
\end{lstlisting}

You can mix and test \*latex and Bibtex/Biber as you wish.
We will probably use Xelatex and Biblatex/Biber for the final document.
This document has been tested with Texlive 2020-2023.

\vspace{\baselineskip}
\noindent
The Overleaf link for this template is:

\url{https://www.overleaf.com/read/vbdzmrfjrntj#c56916}

\vspace{\baselineskip}
\noindent
The Github template link is:

\url{https://github.com/nemo-cluster/10th-bwHPC-Symposium-Freiburg-Template}

%%% END: Document %%%

%%% START: Corresponding Author and ORCID %%%

% Correspondence information, does not have to be main author 
% Select EN or DE
% English
\subsubsection*{Corresponding Author}
% or German
%\subsubsection*{Korrespondenzautor}
\label{sec:cauthor}

Michael Janczyk:~\email{kdrtqhuu@duck.com}\\
eScience Abteilung, Rechenzentrum Albert-Ludwigs-Universität Freiburg,\\
Hermann-Herder-Str. 10, 79104 Freiburg, Deutschland

\subsubsection*{ORCID} % Optional, else remove

Michael Janczyk\,\orcid{https://orcid.org/0000-0003-4886-736X}\\
Dirk von Suchodoletz\,\orcid{https://orcid.org/0000-0002-4382-5104}\\
Bernd Wiebelt\,\orcid{https://orcid.org/0000-0003-2771-4524}

%%% END: Corresponding Author and ORCID %%%

% Print bibliography at end of your document
\printbibliography

\end{document}
